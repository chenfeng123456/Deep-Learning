\documentclass{article}

\usepackage{ctex}


%添加首行缩进,两个字符
\usepackage{indentfirst}
\setlength{\parindent}{2em}

%颜色 用法:\textcolor{name of color}{text}
\usepackage{xcolor}

%照片
\usepackage{graphicx}

%插入python
\usepackage{pythonhighlight}

%并排放置两张图
\usepackage{subfigure}

%引入了一些改进的数学环境,如align
\usepackage{amsmath}

%有序列表
\usepackage{enumerate}

%插入代码
\usepackage{listings}
%插入代码的配置
\definecolor{CPPLight}  {HTML} {686868}
\definecolor{CPPSteel}  {HTML} {888888}
\definecolor{CPPDark}   {HTML} {262626}
\definecolor{CPPBlue}   {HTML} {4172A3}
\definecolor{CPPGreen}  {HTML} {487818}
\definecolor{CPPBrown}  {HTML} {A07040}
\definecolor{CPPRed}    {HTML} {AD4D3A}
\definecolor{CPPViolet} {HTML} {7040A0}
\definecolor{CPPGray}  {HTML} {B8B8B8}
\lstset{
    columns=fixed,    
    breaklines = true,   
    numbers=left,                                        % 在左侧显示行号
    frame=none,                                          % 不显示背景边框
    backgroundcolor=\color[RGB]{245,245,244},            % 设定背景颜色
    keywordstyle=\color[RGB]{40,40,255},                 % 设定关键字颜色
    numberstyle=\footnotesize\color{darkgray},           % 设定行号格式
    commentstyle=\it\color[RGB]{0,96,96},                % 设置代码注释的格式
    stringstyle=\rmfamily\slshape\color[RGB]{128,0,0},   % 设置字符串格式
    showstringspaces=false,                              % 不显示字符串中的空格
    language=Python,                                        % 设置语言
    morekeywords={True,alignas,continute,friend,register,true,alignof,decltype,goto,
    reinterpret_cast,try,asm,defult,if,return,typedef,auto,delete,inline,short,
    typeid,bool,do,int,signed,typename,break,double,long,sizeof,union,case,
    dynamic_cast,mutable,static,unsigned,catch,else,namespace,static_assert,using,
    char,enum,new,static_cast,virtual,char16_t,char32_t,explict,noexcept,struct,
    void,export,nullptr,switch,volatile,class,extern,operator,template,wchar_t,
    const,false,private,this,while,constexpr,float,protected,thread_local,
    const_cast,for,public,throw,std},
    emph={map,set,multimap,multiset,unordered_map,unordered_set,
    unordered_multiset,unordered_multimap,vector,string,list,deque,
    array,stack,forwared_list,iostream,memory,shared_ptr,unique_ptr,
    random,bitset,ostream,istream,cout,cin,endl,move,default_random_engine,
    uniform_int_distribution,iterator,algorithm,functional,bing,numeric,},
    emphstyle=\color{CPPViolet}, 
}




\title{The Fourth Week Report}
\author{Lu Guorui}
\date{2018.7.22}

%改变页边距
\usepackage[a6paper,left=10mm,right=10mm,top=15mm,bottom=15mm]{geometry}


\begin{document}

\maketitle
\renewcommand{\contentsname}{Contents}
\tableofcontents
\newpage


\section{Some Questions Remained}

\subsection{What is Forward and Back propagation}
\indent In short, forward propagation is a way we used to count the value of compound functions. And back propagation is used to count the derivative of compound functions. We needn't complicate them.

\subsection{How can we express the derivative of $\sigma(z)$}
\indent By taking the derivative of $\sigma(z)$, we can easily get the result:$\sigma'(z)=\sigma(z)(1-\sigma)$.
\indent Further, we can gain the consequence:
\begin{align}
dz = \dfrac{\partial L}{\partial z} &= \dfrac{\partial L}{\partial a}\cdot\dfrac{\partial a}{\partial z}  \\
&= (-\frac{y}{a}+\frac{1-y}{1-1}\cdot a(1-a) \\
&= a-y
\end{align}

\begin{align}
dw = \dfrac{\partial L(w,b)}{\partial w}&=\dfrac{\partial L}{\partial z}\cdot\dfrac{\partial z}{\partial w} \\
&=dz\cdot x  \\
&=x(a-y) 
\end{align}

\begin{align}
db = \dfrac{\partial L}{\partial b} &=\dfrac{\partial L}{\partial z}\cdot\dfrac{\partial z}{\partial b}  \\
&= 1\cdot dz \\
&= a-y
\end{align}

\subsection{Some uses of numpy \protect\footnote{The source of package "pythonhighlight":https://github.com/chenfeng123456/python-latex-highlighting}}
\begin{enumerate}
\item Creat a vector:
\begin{python}
x = np.array([...])
\end{python}

\item Call some frequently-used functions
\begin{python}
np.exp(x)
np.log(x)
\end{python}

\item Get the shape of matrices
\begin{python}[language=Python]
x.shape[0] # the number of rows
x.shape[1] # the number of column
\end{python}

\item Change the dimension
\begin{python}
x.reshape((row,col))
\end{python}

\item Count the length of each row
\begin{python}
x_norm = np.linalg.norm(x, axis=1, keepdims = True)
\end{python}

\item Matrix-matrix or matrix-vector multiplication
\begin{python}
# matrix multiplication: 
# x1.shape = (a, m), x2.shape = (m, b)
np.dot(x1,x2) 
# multiply the elements in corresponding locations
# x1.shape = (a, m), x2.shape = (a, m)
np.multiply(x1, x2) 
\end{python}

\item Gain random numbers

\begin{itemize}

\item Get one number
\begin{python}
n = numpy.random.random()
\end{python}

\item Get a matrix
\begin{python}
n = numpy.random.random(size=(3, 2))
\end{python}

\item Generate random numbers between 0 and 1
\begin{python}
np.random.rand(2,3) # (2,3) show the dimension
\end{python}

\item Generate the same random numbers
\begin{python}
'''
Set the same seed every time,
and we can get the same random numbers
'''
numpy.random.seed(num)
numpy.random.rand(the_length_of_the_array)
\end{python}

\end{itemize}


\end{enumerate} 


\end{document}